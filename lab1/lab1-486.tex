% LaTeX template for ECE 486 Lab; this template should be used with "report"
% section in the lab manual

% by Y\"un Han
% 2017-06-06

% A lot of changes have happened during the last four years. So a major overhaul
% is needed.

% -------------------
% Editor + TeX engine
% -------------------
\documentclass{article}
% \usepackage{graphicx}
% \usepackage{caption}
\usepackage{mathtools} % load AMS maths
\usepackage{amsmath}  % for math spacing
\usepackage{amssymb}  % for math spacing
\usepackage[margin=2.5cm]{geometry} % an easy way to change page layout, Thanks to Brady Salz 
\newcommand{\score}{\hfill \underline{\hspace{0.65cm}}\,/} % for score underline
\newcommand\RR{\textsuperscript{\textregistered}~} % for registered mark
\begin{document}
% consider aligning your names like the following
% ----
% ************Lab X Report 
%
% ************by YOUR NAME
% ******Lab Partner: HIS/HER NAME
% ******Lab      TA: Y\"un Han
% ----
\title{\bf Lab \#1 Report\\{\sc Simulation Using \\the Analog Computer}}
\author{Eli Onufrock \\ Section ABA\\
  Lab Partners: Jacob Zora and Josh Powers\\
  Lab TA: Junjie Gao}
\maketitle
\noindent \fbox{\bf \Large Total: \underline{\hspace{0.65cm}}\,/40}
\section*{Question 1 \score 15}
\subsection*{Theoretical and Experimental Results \score 5}
\begin{table}[phtb]
  \begin{center}
    \caption{Theoretical and Experimental Results}
     \label{tbl:lab1_q1}
    \begin{tabular}{c|rr|rr|rr} \hline \hline
      & \multicolumn{2}{c}{$M_p$ (\%)} & \multicolumn{2}{c}{$t_r$ (s)} & \multicolumn{2}{c}{$t_s$ (s)} \\ \cline{2-7} % horizontal line spanning across specified columns
      $\zeta$ & Theory & Experiment & Theory & Experiment & Theory & Experiment \\
      \hline
      2.0 & 0 & 0 & 8.2 & 8.68 & 11.6 & 14.68 \\ 
      1.5 & 0 & 0 & 5.85 & 6.14 & 8.3 & 10.98 \\
      1.0 & 0 & 0 & 3.35 & 3.54 & 5.0 & 7.16 \\
      0.8 & 1.52 & 0.488 & 2.5 & 2.52 & 3.68 & 5.5 \\
      0.7 & 4.6 & 3.71 & 2.16 & 2.14 & 3.02 & 4.96 \\
      0.5 & 16.3 & 15.2 & 1.63 & 1.64  & 6.28 & 7.28 \\
      0.3 & 37.2 & 35.2 & 1.3 & 1.3 & 10.14 & 10.12 \\
      0.2 & 52.7 & 50.8 & 1.21 & 1.2 & 15.08 & 16 \\ \hline 
    \end{tabular} 
    \end{center}
\end{table}
\subsection*{Comparison of Theoretical/Experimental Results \score 5}
% add your discussion here 

\subsection*{Discussion of Variation of $\zeta$ with $M_p$, $t_r$, and $t_s$ \score 5}
(As $\zeta$ decreases, how does $M_p$ change? As $\zeta$ decreases, how about $t_r$ and $t_s$?)
Decreasing $\zeta$ causes $M_p$ to increase, as the reduced damping leads to an increased 
overshoot. 
It also causes $t_r$ to decrease, while $t_s$ decreases as $\zeta$ goes from 2 to about 0.7, 
then increases again, with overdamping increasing the time required to reach the steady state
value to begin with and underdamping causing oscillations before the value settles down.

\section*{Question 2 \score 15}

\subsection*{Effect of $\zeta$ on Pole Locations \score 5}
(Solve for poles in terms of $\zeta$ and sketch a plot of trajectory of pole locations [you can use {\sc Matlab\RR}] when $\zeta$ varies.)
The poles of the transfer function are the roots of $s^2+2\zeta\omega_ns+\omega^2_n$,
which can be found with the quadratic formula: 

\begin{center}
	$\frac{-2\zeta\omega_n \pm \sqrt{(2\zeta\omega_n)^2-4(1)\omega^2}}{2}$
	$ = -\zeta\omega_n \pm \omega_n\sqrt{\zeta^2-1}$
\end{center}
which, since $\omega_n$ = 1, is: 

\begin{center}
	$\zeta \pm \sqrt{\zeta^2-1}$
\end{center}

\begin{figure}
	\includegraphics[scale=0.6]{pole-locations.png}
	\centering
	\caption{Location of poles of transfer function in complex plane for varying values of 
		$\zeta$}
	\label{poles}

\end{figure}
\subsection*{Effect of Pole Locations on $M_p$, $t_r$, and $t_s$ for an Underdamped System \score 5}
(What is the value of $\zeta$ when a system is \emph{underdamped}? As $\zeta$ increases, what happens to $M_p$, $t_r$, and $t_s$?)


When $\zeta < 1$, the system is underdamped. For lower values of $\zeta$, the system has 
greater overshoot ($M_p$)and more oscillation, leading to a longer settling time ($t_s$),
but a faster rise time ($t_r$). 
As $\zeta$ increases, rise time increases, but overshoot and settling time decrease.

\subsection*{Effect of Pole Locations on $M_p$, $t_r$, and $t_s$ for an Overdamped/Critically Damped System \score 5}
(What is the value of $\zeta$ when a system is \emph{overdamped}? \emph{Critically damped}? And as $\zeta$ increases, what happens to $M_p$, $t_r$, and $t_s$?)
% add your discussion here
The system becomes critically damped when $\zeta$ = 1. When $\zeta$ has increased to this point,
there ceases to be overshoot and there is no oscillation, so the settling time is low. 
(The rise time is increased from underdamped conditions, so some underdamped ratios may cause
the system to reach the acceptable range for settling time faster as seen in the results.)

As $\zeta$ keeps increasing, there continues to be no overshoot, but rise time also increases,
causing the settling time to increase. Under these conditions the system is overdamped.

\section*{Question 3 \score 10}

\subsection*{Comparison of $2^{\rm nd}$ Order System with $1^{\rm st}$ Order System with Dominant Pole \score 6}
(What are the similarities/differences between the response of an overdamped $2^{\rm nd}$ order system to the response of a $1^{\rm st}$ order system with the \emph{dominant} (less negative, closer to the origin) pole of the $2^{\rm nd}$ order's poles?)
% add your discussion here

\subsection*{Effect of $\zeta$ on Accuracy of Approximation \score 4}
(What is the effect of magnitude of $\zeta$ on the accuracy of the approximations? Also include your graphs.)
% add your discussion here
\newline \\[10mm]
\noindent {\bf \large Attachments}
\begin{itemize}
\item Plots obtained during lab
\item Sample response with relevant points for calculating $M_p$ , $t_s$ and $t_r$ marked
\item Step responses comparing $2^{\rm nd}$ order systems and $1^{\rm st}$ order approximations
\item {\sc Matlab} code
\end{itemize}

\end{document}
