\documentclass[letterpaper,11pt]{article}
\usepackage[T1]{fontenc}
\usepackage[utf8]{inputenc}
\usepackage[english]{babel}
\usepackage[left=1.25in,right=1.25in]{geometry}
\usepackage{lmodern}
\usepackage{amsmath}
\usepackage{amsfonts}
\usepackage{amssymb}
\usepackage{amsthm}
\usepackage{graphicx}
\usepackage{color}
\usepackage{xcolor}
\usepackage{url}
\usepackage{textcomp}
\usepackage{parskip}

\title{Prelab 2: Digital Simulation}
\author{Elijah Onufrock}
\date{\today}

\begin{document}

\maketitle

\section{All-Integrator Block Diagram}
$H_1(s) = \frac{Y(s)}{U(s)} = \frac{25}{s^2+6s+25}$

This can be rearranged to give the equation
$s^2Y(s) = 25U(s) - 6sY(s) - 25Y(s)$
, which produces the following all-integrator block diagram:

\begin{figure}[h]
	\centering
	\includegraphics[width=\linewidth]{prelab-block.png}
	\caption{All-integrator Block Diagram for $H_1(s)$}
	\label{fig:block}
\end{figure}

This system has $\omega_n = \sqrt{25} = 5$ and 
$\zeta = \frac{6}{2\omega_n} = \frac{6}{10} = 0.6 $,
indicating the system is underdamped.
Its poles are at the zeros of $s^2+6s+25$, which are $-3+j4$ and $-3-j4$.

\end{document}
